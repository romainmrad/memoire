\Chapter{ANALYSE DES RÉSULTATS}\label{sec:Theme2}

Après avoir déterminé des expressions analytiques pour les différentes fonctions étudiées, il est essentiel d'analyser les résultats obtenus afin de vérifier leur validité et leur cohérence avec le modèle théorique. 

Pour évaluer le comportement des fonctions obtenues, il est nécessaire de se placer dans un cadre particulier du processus étudié. En se basant sur la définition du processus \acs{CIR} (\ref{cir_eq}) et du problème de premier passage (\ref{fpt_definition}), les paramètres suivants sont considérés pour l'ensemble des analyses:

\begin{itemize}
    \item Vitesse de retour à la moyenne du \acs{CIR}: $a=0.1$
    \item Niveau moyen du \acs{CIR}: $b=0.9$
    \item Volatilité instantanée du \acs{CIR}: $\sigma=1$
    \item Frontière supérieure pour le \acs{TPP}: $c=1$
\end{itemize}

\section{Diffusion pure}

D'abord, les résultats trouvés pour le processus de diffusion pure sont abordés.

\subsection{Commande Optimale Stochastique}

Enfin, il est nécessaire de valider les expressions obtenues dans le cadre des trois problèmes de commande optimale étudiés. Pour cela, il convient de tracer la fonction valeur $F(x)$ ainsi que le contrôle optimal $u^*(x)$, et s'il est possible, pour plusieurs valeurs des différents paramètres des coûts. Afin d'isoler l'influence d'un paramètre lors de l'analyse, les autres seront fixés à $1$.

\subsubsection{Problème linéarisable 1 \textemdash~P1}\phantom{}\\

\FloatBarrier\subsubsection{Problème linéarisable 2 \textemdash~P2}\phantom{}\\

\subsubsection{Problème non linéarisable \textemdash~P3}\phantom{}\\
