% Résumé du mémoire.
% Abstract in French.
%
\chapter*{RÉSUMÉ}\thispagestyle{headings}
\addcontentsline{toc}{compteur}{RÉSUMÉ}

Le processus de diffusion de \ac{CIR} est défini par l'équation différentielle stochastique
\[
dX(t) = a[b - X(t)]\,dt + \sigma \sqrt{X(t)}\,dW(t)
\]
où \(W(t)\) est un mouvement brownien standard. Cette étude porte sur le moment de premier passage
\[
\tau(x) = \inf\{ t \geq 0 : X(t) \notin (0, c) \mid X(0) = x \in (0, c) \}
\]
c'est-à-dire le premier instant où le processus quitte l'intervalle \((0, c)\). Des expressions explicites de la fonction génératrice des moments, de l'espérance de \(\tau(x)\) et de l'aire moyenne parcourue par le processus sont obtenues. De plus, un problème de commande optimale est étudié. En outre, des sauts sont ajoutés au processus afin de caractériser la probabilité de toucher la frontière inférieure ainsi que le temps moyen pour quitter l'intervalle. Enfin, le dépassement moyen au dessus de la frontière $c$ est analysé. 


