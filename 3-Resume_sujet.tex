% Résumé du mémoire.
% Abstract in French.
%
\chapter*{RÉSUMÉ}\thispagestyle{headings}
\addcontentsline{toc}{compteur}{RÉSUMÉ}

Le processus de diffusion de Cox-Ingersoll-Ross (CIR) est défini par l'équation différentielle stochastique suivante:
\[
dX(t) = a[b - X(t)]\,dt + \sigma \sqrt{X(t)}\,dW(t)
\]
où \(W(t)\) désigne un mouvement brownien standard. Ce processus est largement utilisé en finance, notamment pour modéliser les taux d'intérêt ou la volatilité stochastique, en raison de sa propriété de rester strictement positif sous certaines conditions sur les paramètres.

L'analyse menée dans cette étude porte sur le temps de premier passage:
\[
\tau(x) = \inf\{ t \geq 0 : X(t) \notin (0, c) \mid X(0) = x \in (0, c) \}
\]
qui correspond au premier instant où la trajectoire de \(X(t)\) sort de l'intervalle \((0,c)\). Trois quantités principales sont caractérisées analytiquement dans ce cadre : la fonction génératrice des moments du temps de sortie \( \mathds{E}[e^{-\alpha \tau(x)}] \), l'espérance du temps de sortie \( \mathds{E}[\tau(x)] \), et l'aire moyenne sous la trajectoire du processus jusqu'à l'instant de sortie.

Une extension du modèle est ensuite considérée, dans laquelle des sauts sont ajoutés à la dynamique continue du processus. Ce cadre permet d'étudier plus finement certains phénomènes asymétriques, tels que la probabilité de sortie par la borne inférieure \(0\) dans le cas de sauts orientés vers le bas, ou le dépassement moyen de la borne supérieure \(c\) en présence de sauts positifs. L'effet de ces sauts sur le temps moyen de sortie est également examiné.

Enfin, plusieurs problèmes de commande optimale sont étudiées dans le cadre purement diffusif. Ils consistent à déterminer la politique de contrôle optimale minimisant un coût attendu, ainsi que la fonction valeur associée. Ces problème s'inscrivent dans la classe des \textit{homing problems}, où l'objectif est de guider le processus vers une cible ou de réguler sa trajectoire de manière optimale.