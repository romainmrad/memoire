%%
%%  Annexes
%%
%%  Note: Ne pas modifier la ligne ci-dessous. / Do not modify the following line.
\phantomsection
\ifthenelse{\equal{\Langue}{english}}{
	\addcontentsline{toc}{compteur}{APPENDICES}
}{
	\addcontentsline{toc}{compteur}{ANNEXES}
}
%%
%%
%%  Toutes les annexes doivent être inclues dans ce document
%%  les unes à la suite des autres.
%%  All annexes must be included in this document one after the other.



\Annexe{Fonctions Spéciales}\label{special_functions}
Cette annexe contient une définition de toutes les fonctions spéciales utilisées dans ce mémoire. Pour plus d'information, voir~\cite{NIST:DLMF}.
\section*{Fonctions Gamma}
\begin{itemize}
    \item La fonction Gamma est définie par:
    \[
        \begin{aligned}
            \Gamma(z)&:=\int_0^{+\infty}t^{z-1}e^{-t}dt \\
            \Gamma(n)&:=(n-1)!
        \end{aligned}
    \]
    \item Les fonction Gamma incomplètes sont définies par:
    \[
    \begin{aligned}
        \Gamma(s,x)&:=\int_x^{+\infty}t^{s-1}e^{-t}dt \\
        \gamma(s,x)&:=\int_0^x t^{s-1}e^{-t}dt
    \end{aligned}
    \] 
\end{itemize}

\section*{Symbole de Pochhammer}
Le symbole de Pochhammer (ou factoriel ascendant) est défini par:
\[
{(a)}_n:=\prod_{i=1}^n (a+i-1)
\]

\section*{Fonctions hypergéométriques}
\begin{itemize}
    \item Les fonctions hypergéométriques confluentes de première et seconde espèce (aussi appelées fonctions de Kummer et Tricomi) sont définies par:
    \[
        \begin{aligned}
            \Phi(s, t, z) &:= \sum_{n=0}^{\infty} \frac{{(s)}_n \, z^n}{{(t)}_n \, n!}  \\
            \Psi(s,t,z) &:= \frac{\Gamma(1-t)}{\Gamma(s+1-t)}\Phi(s,t,z)+\frac{\Gamma(t-1)}{\Gamma(s)}z^{1-t}\Phi(s+1-t,2-t,z)
        \end{aligned}
    \]
    \item La fonction hypergéométrique généralisée est définie par:
    \[
    _p F_q\left([s_1,\dots s_p],\,[t_1\dots t_q],\,z\right):=\sum_{n=0}^\infty\frac{{(s_1)}_n\cdots{(s_p)}_n}{{(t_1)}_n\cdots {(t_q)}_n}\frac{z^n}{n!}
    \]
    Il est intéressant de noter que la fonction hypergéométrique confluente de première espèce $\Phi(\cdot,\cdot,\cdot)$ n'est autre que le cas particulier $_1F_1(\cdot,\cdot,\cdot)$.
\end{itemize}

\section*{Fonction intégrale exponentielle généralisée}
\[
E_n(x):=\int_1^{+\infty}\frac{e^{xt}}{t^n}dt
\]

\section*{Fonctions de Bessel}
\begin{itemize}
    \item Les fonctions de Bessel de première et seconde espèce sont définies par:
    \[
    \begin{aligned}
        J_\alpha(x)&:=\sum_{m=0}^{+\infty}\frac{(-1)^m}{m!\Gamma(m+\alpha+1)}{\left(\frac{x}{2}\right)}^{2m+\alpha} \\
        Y_\alpha(x)&:=\frac{J_\alpha(x)\cos(\alpha\pi)-J_{-\alpha}(x)}{\sin(\alpha\pi)}
    \end{aligned}
    \]
    \item Les fonctions de Bessel modifiées de première et seconde espèce sont définies par:
    \[
    \begin{aligned}
        I_\alpha(x)&:=i^{-\alpha}J_\alpha(ix)\\
        K_\alpha(x)&:=\frac{\pi}{2}\frac{I_{-\alpha}(x)-I_\alpha(x)}{\sin(\alpha\pi)}
    \end{aligned}
    \]
\end{itemize}

\Annexe{Générateur Infinitésimal}\label{infinitesimal_generator}

Le générateur infinitésimal \( \mathcal{L} \) associé à un processus \( \{X(t), t\geq0\} \) est défini, pour toute fonction \( f \in C^2(\mathds{R}) \), par:
\[
    \mathcal{L}f(x) = \lim_{t \to 0} \frac{\mathds{E}_x[f(X(t))] - f(x)}{t}
\]
Cet opérateur permet d'établir des équations différentielles décrivant l'évolution d'une statistique du processus étudié. Pour plus d'informations, voir~\cite{bakry2014}.

\section*{Diffusion pure}

Soit \( \{X(t),\, t \geq 0\} \) un processus de diffusion pure à valeurs réelles, solution forte de l'équation différentielle stochastique:
\[
    dX(t) = \mu(t,X(t))\,dt + \sigma(t,X(t))\,dW(t)
\]
Sous des hypothèses standards d'unicité trajectorielle, le générateur infinitésimal s'écrit:
\[
    \mathcal{L}f(x) = \mu(x) f'(x) + \frac{1}{2} \sigma^2(x) f''(x)
\]

\section*{Diffusion avec sauts}

Soit \( \{X(t),\, t \geq 0\} \) un processus de diffusion avec sauts à valeurs réelles, solution forte de l'équation différentielle stochastique:
\[
    dX(t) = \mu(t, X(t))\,dt + \sigma(t, X(t))\,dW(t) + \int_0^t\int_{\mathds{R}} yN(ds,dy)
\]
Sous des hypothèses standards d'unicité trajectorielle, le générateur infinitésimal s'écrit:
\[
    \mathcal{L}f(x) = \mu(x) f'(x) + \frac{1}{2} \sigma^2(x) f''(x) + \int_{\mathds{R}} \left[f(x+y) - f(x)\right] \nu(dy)
\]

\Annexe{Unicité Trajectorielle}\label{trajecotry_uniqueness}

\section*{Diffusion pure}

Soit un processus \( \{X(t),\, t \geq 0\} \) à valeurs dans \(\mathds{D}\) défini par une certaine \ac{EDS}:
\begin{equation}\label{diffusion_sde}
    dX(t) = \mu(t,X(t))\,dt + \sigma(t,X(t))\,dW(t)
\end{equation}
Si les conditions suivantes sont satisfaites:
\begin{itemize}
    \item \textbf{Dérive Lipschitz}: il existe une constante \( K > 0 \) telle que, pour tout \( x,y \in \mathds{D} \),
    \[
    |\mu(x)-\mu(y)|\leq K|x-y|
    \]
    \item \textbf{Diffusion Hölder}: il existe des constantes \( K > 0 \) et \(\alpha\in[\frac{1}{2},1)\) telles que, pour tout \( x,y \in \mathds{D} \),
    \[
    |\sigma(x)-\sigma(y)|\leq K{|x-y|}^\alpha
    \]
    \item \textbf{Croissance au plus linéaire}: il existe une constante \( K > 0 \) telle que, pour tout \( x \in \mathds{D} \),
    \[
    |\mu(x)|^2 + |\sigma(x)|^2 \leq K(1 + |x|^2)
    \]
\end{itemize}
$X(t)$ est alors l'unique solution forte, càdlàg, non-explosive  et adaptée de l'équation (\ref{diffusion_sde}).

\section*{Diffusion avec sauts}
Soit un processus \( \{X(t),\, t \geq 0\} \) défini par une certaine \ac{EDS}:
\begin{equation}\label{jump_diffusion_sde}
    dX(t) = \mu(t, X(t))\,dt + \sigma(t, X(t))\,dW(t) + \int_0^t\int_{\mathds{R}} yN(ds,dy)
\end{equation}
Si les conditions d'unicité trajectorielle associées à la diffusion sont satisfaites (dérive Lipschitz, diffusion Hölder, croissance au plus linéaire), et si la mesure \( \nu \), modélisant la loi des tailles des sauts, vérifie la \textbf{Condition de Lévy minimale}:
\[
\int_{\mathds{R}}\min(1, y^2)\, \nu(dy) < \infty
\]
$X(t)$ est alors l'unique solution forte, càdlàg, non-explosive et adaptée de l'équation (\ref{jump_diffusion_sde}).

\Annexe{Simulations en Commande Optimale}\label{control_simulations}

Les données suivantes découlent de l'analyse de 1000 simulations différentes du \acs{CIR}. 

\section*{Accélération de sortie}
Les longueurs moyennes des trajectoires avec et sans contrôle simulées pour chaque problème sont présentées dans le tableau ci-dessous.
\begin{table}[htb]
        \centering
        \caption{Longueurs moyennes des trajectoires simulées}\label{tab:simulation_lengths}
        \renewcommand{\arraystretch}{1.4}
        \begin{tabular}{||c|c|c||}
        \hline
        Problème &
        Longueur moyenne CIR non contrôlé &
        Longueur moyenne CIR contrôlé \\
        \hline\hline
        P1 & 6904.4004 & 2595.1005 \\
        %P2 & 7539.4178 & 2802.1509 \\
        P2 & 7004.5290 & 6576.0133 \\
        \hline
        \end{tabular}
\end{table}\FloatBarrier Ces dernières sont mesurées en discrétisant la trajectoire avec des pas temporels de longueur $10^{-4}$. Il est donc possible d'estimer un facteur d'accélération moyenne pour le temps de sortie comme suit: 
\[
\text{Accélération}=\frac{\text{Longueur moyenne CIR non contrôlé}}{\text{Longueur moyenne CIR contrôlé}}
\]

\section*{Statistiques de sortie}
La fréquence de sortie par chaque frontière peut être estimée pour chaque configuration étudiée:
\begin{table}[htb]
    \centering
    \caption{Fréquence de sortie des trajectoires simulées}\label{tab:simulation_exit_frequency}
    \renewcommand{\arraystretch}{1.4}
    \begin{tabular}{||c|c|c||}
        \hline
        Configuration & Fréquence de Sortie en 0 & Fréquence de Sortie en $c=1$ \\
        \hline\hline
        Sans contrôle & 44.4\,\% & 55.6\,\% \\
        P1 & 33\,\% & 67\,\% \\
        %P2 & 66.7\,\% & 33,3\,\% \\
        P2 & 45.2\,\% & 54.8\,\% \\
        \hline
    \end{tabular}
\end{table}