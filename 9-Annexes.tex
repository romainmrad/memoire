% %%
% %%  Annexes
% %%
% %%  Note: Ne pas modifier la ligne ci-dessous. / Do not modify the following line.
% \ifthenelse{\equal{\Langue}{english}}{
% 	\addcontentsline{toc}{compteur}{APPENDICES}
% }{
% 	\addcontentsline{toc}{compteur}{ANNEXES}
% }
% %%
% %%
% %%  Toutes les annexes doivent être inclues dans ce document
% %%  les unes à la suite des autres.
% %%  All annexes must be included in this document one after the other.
% \Annexe{Démo}
% Texte de l'annexe A\@. Remarquez que la phrase précédente se termine
% par une lettre majuscule suivie d'un point. On indique explicitement
% cette situation à \LaTeX{} afin que ce dernier ajuste correctement
% l'espacement entre le point final de la phrase et le début de la
% phrase suivante.


% \begin{landscape}
% \Annexe{Encore une annexe / Another Appendix}
% Texte de l'annexe B\@ en mode «landscape».
% \end{landscape}

% \Annexe{Une dernière annexe / The Last Appendix}
% Texte de l'annexe C\@.

\Annexe{Fonctions spéciales}\label{special_functions}

Cette annexe contient une définition de toutes les fonctions spéciales utilisées dans ce mémoire. Pour plus d'information, voir\cite{NIST:DLMF}.
\paragraph{Fonctions Gamma}
\begin{itemize}
    \item La fonction Gamma est définie par:
    \[
        \begin{aligned}
            \Gamma(z)&:=\int_0^{+\infty}t^{z-1}e^{-t}dt \\
            \Gamma(n)&:=(n-1)!
        \end{aligned}
    \]
    \item Les fonction Gamma incomplètes sont définies par:
    \[
    \begin{aligned}
        \Gamma(s,x)&:=\int_x^{+\infty}t^{s-1}e^{-t}dt \\
        \gamma(s,x)&:=\int_0^x t^{s-1}e^{-t}dt
    \end{aligned}
    \] 
\end{itemize}

\paragraph{Symbole de Pochhammer}
\[
{(a)}_n:=\prod_{i=1}^n (a+i-1)
\]

\paragraph{Fonctions hypergéométriques}
\begin{itemize}
    \item Les fonctions hypergéométriques confluentes de première et seconde espèce (aussi appelées fonctions de Kummer et Tricomi) sont définies par:
    \[
        \begin{aligned}
            \Phi(s, t, z) &:= \sum_{n=0}^{\infty} \frac{{(s)}_n \, z^n}{{(t)}_n \, n!}  \\
            \Psi(s,t,z) &:= \frac{\Gamma(1-t)}{\Gamma(s+1-t)}\Phi(s,t,z)+\frac{\Gamma(t-1)}{\Gamma(s)}z^{1-t}\Phi(s+1-t,2-t,z)
        \end{aligned}
    \] 
    \item La fonction hypergéométrique généralisée est définie par:
    \[
    _p F_q\left([s_1,\dots s_p],\,[t_1\dots t_q],\,z\right):=\sum_{n=0}^\infty\frac{{(s_1)}_n\cdots{(s_p)}_n}{{(t_1)}_n\cdots {(t_q)}_n}\frac{z^n}{n!}
    \]
    Il est intéressant de noter que la fonction hypergéométrique confluente de première espèce $\Phi(\cdot,\cdot,\cdot)$ n'est autre que le cas particulier $_1F_1(\cdot,\cdot,\cdot)$.
\end{itemize}

\paragraph{Fonction intégrale exponentielle généralisée}
\[
E_n(x):=\int_1^{+\infty}\frac{e^{xt}}{t^n}dt
\]

\paragraph{Fonctions de Bessel}
\begin{itemize}
    \item Les fonctions de Bessel de première et seconde espèce sont définies par:
    \[
    \begin{aligned}
        J_\alpha(x)&:=\sum_{m=0}^{+\infty}\frac{(-1)^m}{m!\Gamma(m+\alpha+1)}{\left(\frac{x}{2}\right)}^{2m+\alpha} \\
        Y_\alpha(x)&:=\frac{J_\alpha(x)\cos(\alpha\pi)-J_{-\alpha}(x)}{\sin(\alpha\pi)}
    \end{aligned}
    \]
    \item Les fonctions de Bessel modifiées de première et seconde espèce sont définies par:
    \[
    \begin{aligned}
        I_\alpha(x)&:=i^{-\alpha}J_\alpha(ix)\\
        K_\alpha(x)&:=\frac{\pi}{2}\frac{I_{-\alpha}(x)-I_\alpha(x)}{\sin(\alpha\pi)}
    \end{aligned}
    \]
\end{itemize}

