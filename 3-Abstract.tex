%% Abstract
%%
%% Traduction anglaise fidèle et de qualité du résumé de la recherche écrit en français et non une traduction littérale. 
%%

\chapter*{ABSTRACT}\thispagestyle{headings}
\addcontentsline{toc}{compteur}{ABSTRACT}
%
\begin{otherlanguage}{english}
This work focuses on first-passage-time properties of the Cox-Ingersoll-Ross (CIR) diffusion, a stochastic process governed by the equation:
\[
dX(t) = a[b - X(t)]\,dt + \sigma \sqrt{X(t)}\,dW(t),
\]
where \(W(t)\) denotes a standard Brownian motion. The quantity of interest is the exit time from the interval \((0, c)\), defined as:
\[
\tau(x) = \inf\{ t \geq 0 : X(t) \notin (0, c) \mid X(0) = x \in (0, c) \}.
\]
Analytical expressions are obtained for the Laplace transform of the exit time \( \mathds{E}[e^{-\alpha \tau(x)}] \), its expectation \( \mathds{E}[\tau(x)] \), and the average area under the path prior to exit.

The framework is extended to incorporate jumps into the dynamics. This generalization enables the study of the probability of hitting the lower boundary under downward jumps, the expected overshoot beyond the upper boundary in the presence of upward jumps, and the modification of the mean exit time due to the jump component.

Optimal control problems are also considered in the purely diffusive setting. The associated value function and the optimal control minimizing a given cost functional are characterized. This setup corresponds to a homing-type control problem, where the trajectory is influenced to reach or remain within a target region under optimal intervention.

\end{otherlanguage}
