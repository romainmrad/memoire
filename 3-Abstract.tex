%% Abstract
%%
%% Traduction anglaise fidèle et de qualité du résumé de la recherche écrit en français et non une traduction littérale. 
%%

\chapter*{ABSTRACT}\thispagestyle{headings}
\addcontentsline{toc}{compteur}{ABSTRACT}
%
\begin{otherlanguage}{english}

The \ac{CIR} diffusion process is defined by the stochastic differential equation
\[
dX(t) = a[b - X(t)]\,dt + \sigma \sqrt{X(t)}\,dW(t)
\]
where \(W(t)\) is a standard Brownian motion. This study focuses on the first passage time
\[
\tau(x) = \inf\{ t \geq 0 : X(t) \notin (0, c) \mid X(0) = x \in (0, c) \}
\]
that is, the first time the process exits the interval \((0, c)\). Explicit expressions for the moment-generating function, the expected value of \(\tau(x)\) and the average area under the process are derived. In addition, a stochastic control problem is studied. Furthermore, jumps are added to the process with the aim to analyse the probability of hitting the lower boundary first and the mean exit time. Finally, the average overshoot above the $c$ frontier is characterised.


\end{otherlanguage}
